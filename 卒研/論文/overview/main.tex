\documentclass[a4jsme]{jsmepaper}
\usepackage{fancyhdr}
\usepackage[dvipdfmx]{graphicx}
\usepackage{fancyhdr}
\usepackage{here}
\usepackage{amsmath}
\usepackage[subrefformat=parens]{subcaption}

\renewcommand{\figurename}{Fig.}
%%% dummy.tex
%
\DeclareFontShape{JY1}{mc}{m}{it}{<5> <6> <7> <8> <9> <10> sgen*min
    <10.95><12><14.4><17.28><20.74><24.88> min10
<-> min10}{}
\DeclareFontShape{JT1}{mc}{m}{it}{<5> <6> <7> <8> <9> <10> sgen*tmin
    <10.95><12><14.4><17.28><20.74><24.88> tmin10
<-> tmin10}{}
\DeclareFontShape{JY1}{mc}{bx}{it}{<5> <6> <7> <8> <9> <10> sgen*min
    <10.95><12><14.4><17.28><20.74><24.88> min10
<-> min10}{}
\DeclareFontShape{JT1}{mc}{bx}{it}{<5> <6> <7> <8> <9> <10> sgen*tmin
    <10.95><12><14.4><17.28><20.74><24.88> tmin10
<-> tmin10}{}

\endinput
%%%% end of dummy.tex

\lhead{\normalsize{2016年度卒業論文概要}} % 年度によって適宜変更
\rhead[]{}
\renewcommand{\headrulewidth}{0.0pt}

\newcommand{\bhline}[1]{\noalign{\hrule height #1}}
%
\jtitle{日本語タイトル}
\etitle{English title}

\jauthor{学番 氏名}
\eauthor{full name}

\jaffiliation{
    & \hspace{-1cm}\normalsize{指導教員:林原靖男}
}

\abstract{
In this paper, aaaaaaaaaaaaaaaaaaaa
}
\keywords{RDC LAB, Yasuo Hayashibara, template}

\begin{document}
\maketitle
\thispagestyle{fancy}
\small



%%%%%%%%%%%%%%%%%%%%%%%%%%%%%%%%%%%%%%%%%%%%
%%%  本文
%%%%%%%%%%%%%%%%%%%%%%%%%%%%%%%%%%%%%%%%%%%%
\section{諸元}
本文を書いていく


% 参考文献リストを使わない場合はこんな感じで追加していく
\scriptsize
\begin{thebibliography}{9}
  \bibitem{Hentschel} M. Hentschel, O. Wulf and B. Wagner: “A gps and laser-based localization for urban and
non-urban outdoor environments”, IEEE (2008).
  \bibitem{Yamazaki} 山崎 , 竹内 , 大野 , 田所: “ 三次元地形情報および GPS を用いたパーティクルフィルタによるマル
チパスを考慮した自己位置推定 ”, 日本ロボット学会誌 , vol.29, pp. 702–709 (2011).
の基礎と実力”, トランジスタ技術2016年2月号, 53(2), pp. 66–79 (2016)
\end{thebibliography}
\small

\end{document}
